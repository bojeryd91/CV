\documentclass{clean_CV}

% Add a BibTeX-style file encoding all of your publications to include here. You can export this from Zotero. Only include
% publications you want to appear here!
\addbibresource{publications.bib}

\usepackage{hyperref}
\usepackage{multicol}

\author{Jesper Böjeryd}
%\renewcommand{\today}{November 20, 2023}

\newcommand{\datetabspace}{4.5em}

\begin{document}

\maketitle
% In this section, you can use any of the FontAwesome icons. The commands \faCenter and \faCenterStyle have been defined to properly center the icons
% when using the default font settings.
%
% You can use any of the icons listed in the fontawesome5 package documentation (https://ctan.math.utah.edu/ctan/tex-archive/fonts/fontawesome5/doc/fontawesome5.pdf)
% If you need to specify a specific style (as is done here for the address card), you should use the two-argument \faCenterCycle command

%\section{Contact information}
%\vspace{-0.5em}
\begin{center}
\begin{tabular}{clp{.15\textwidth}cl}
    \faMapMarker & 765 Weyburn Terrace, Apt. 225 && \faEnvelopeO & bojeryd91@gmail.com \\
    & Los Angeles, CA 90024 && \faPhone & +1 (310) 775-3781\\
    & USA && \faPaperclip & \url{www.jesperbojeryd.se} \\
    \faGlobe & Swedish citizen && \faGithub & \url{https://github.com/bojeryd91}
\end{tabular}
\end{center}
\vspace{-1.5em}

\section{Research and teaching interests}
    Macroeconomics, household finance, migration, housing, durable consumption, monetary policy, computational methods
\section{Education}

% The datetabular environment takes one argument, which is the width of the left date column. As seen here:
%   9em is a good choice for "dual-date" formats (e.g. Sep 2015 - Nov 2019).
%   4em is a good choice for month/year dates (Sep 2014).
%   2em is a good choice for year-only dates (as seen in the publications)
\begin{datetabular}{\datetabspace}
% This is just a tabular environment, for the most part. The dateentry command has been defined for
% convenience. It takes two arguments, the first is the date and the second is whatever you wish placed to the right.
\dateentry{2018--Present}{
\textbf{Ph.D. in Economics, University of California, Los Angeles, USA}

%C.Phil., M.A.

References:\vspace{6pt}

{\renewcommand{\tabcolsep}{24pt}}
\begin{tabular}{p{7cm}p{7cm}}
    \href{https://www.leeohanian.com/}{\textbf{Lee Ohanian (Main advisor)}} \newline
    Department of Economics, UCLA \newline
    ohanian@econ.ucla.edu
    &
    \href{https://roinevestman.com/}{\textbf{Roine Vestman}} \newline
    Department of Economics, \newline
    Stockholm University \newline
    roine.vestman@su.se
    \\[0.5em]
    \href{https://sites.google.com/site/andyatkeson/}{\textbf{Andy Atkeson}} \newline
    Department of Economics, UCLA \newline
    andy.atkeson@gmail.com
    &
    \href{https://sites.google.com/site/pierreolivierweill/}{\textbf{Pierre-Olivier Weill}} \newline
    Department of Economics, UCLA \newline
    poweill@gmail.com
\end{tabular}}

\dateentry{2010--2015}{\textbf{Royal Institute of Technology (KTH), Sweden}

M.Eng., M.S., B.S. in Engineering Physics/Applied and Computational Mathematics
}


%\dateentry{Mar--Jun 2013}{\textbf{Korean Advanced Institute of Science and Technology, Republic of South Korea}

%Exchange semester. Studied individual courses and wrote my bachelor thesis.}

%\dateentry{Spring 2010}{\textbf{Karlstad University, Sweden}

%Course: Elementary Algebra. This was a unique one-year project to encourage talented students to study college-level mathematics during my senior year in high school.}
\end{datetabular}


\section{Research papers}

\begin{datetabular}{\datetabspace}
% This is just a tabular environment, for the most part. The dateentry command has been defined for
% convenience. It takes two arguments, the first is the date and the second is whatever you wish placed to the right.
\dateentry{JMP}{%
\href{https://www.jesperbojeryd.se/jmp}{\textbf{Should I stay or should I go? The role of housing in understanding limited
inter-regional worker mobility}} (\href{https://www.jesperbojeryd.se/jmp}{Latest version can be found at \texttt{www.jesperbojeryd.se/jmp}})

Our understanding why workers are slow to leave places in economic decline is limited but is important in the design of policy. Using Norwegian administrative data on workers in the oil-producing Stavanger region and the 2014 plunge in global oil prices, I empirically document that workers' housing wealth is a key variable in explaining who stays or goes. I show that workers with little or no housing wealth left the region at a higher rate following the shock while homeowners with more housing wealth were more likely to stay. The richness of the data allows me to control for potentially confounding factors and selection into housing tenure. Using a life-cycle model with endogenous location, housing, and saving choices as well as home prices, I show that the value of moving is reduced for homeowners when home prices fall in response to a negative labor demand shock and that they are worse off by staying compared to renters. Renters are compensated by a rent reduction. On net, the difference in locations' present values for homeowners increases less than for renters, who therefore move more. The model also shows that moving subsidies are more effective at stimulating renter migration, indicating that untargeted subsidies do not reach the worst impacted workers.}

\dateentry{Working papers}{
\href{https://roinevestman.com/wp-content/uploads/2023/03/DP18034-compressed.pdf}{\textbf{The housing wealth effect: Quasi-experimental evidence}}, \emph{with Dany Kessel, Björn Tyrefors, and Roine Vestman} (\href{https://roinevestman.com/wp-content/uploads/2023/03/DP18034-compressed.pdf}{\underline{Link}})


%We estimate the housing wealth effect on car consumption by using an unexpected announcement of the continuation of Bromma Airport outside Stockholm. Affected households lost 19\% house wealth relative to households outside the airport noise zone, and we find a reduction in the price of new cars affected households buy in the following year and no change on the extensive margin. By studying credit use, we find that the reduction in equity-withdrawal capacity is the driving force rather than changes in lifetime wealth. Our findings are consistent with a calibrated life-cycle model.

\smallskip

\href{http://www.adambaybutt.org/uploads/1/2/4/9/124972193/baybutt_bojeryd-2021-qe_and_firms.pdf}{\textbf{Heterogeneous effects of QE on corporate bonds and firm outcomes}, \emph{with Adam Baybutt} (\href{http://www.adambaybutt.org/uploads/1/2/4/9/124972193/baybutt_bojeryd-2021-qe_and_firms.pdf}{\underline{Link}})}
%Since the Financial Crisis, quantitative easing has become a standard monetary policy tool. We ask how the Federal Reserve's QE programs during the 2010--2020s heterogeneously affected public firms' borrowing costs in the corporate bond market and if it influenced real outcomes. Results are ambiguous and indicate very small effects where dividends stand out to be most positively affected.
}

\dateentry{Work in progress}{
\textbf{What do 12 billion card transactions say about house prices and consumption?}, \emph{with Knut Are Aastveit, Magnus Gulbrandsen, Ragnar Juelsrud, and Kasper Roszbach}

%Using the near-universe of digital transactions, we study how public sector employees and retirees responded to falls in home prices in Stavanger, Norway, compared to similar households in other parts of Norway following the oil price plunge of 2014--2016. We find overall small effects on consumption, which are concentrated to durable expenditures such as furniture and car purchases. The effects are also greater for more indebted households.

\smallskip

\textbf{Durable expenditures, indebtedness, and monetary policy}

%Through a full-population panel for Sweden 1995--2015, I study how monetary policy affects new and used car purchases. To be extended with housing transactions.

\smallskip

\textbf{Measuring household expenditures: Digital transactions versus imputed expenditures}, \emph{with Knut Are Aastveit, Andreas Fagereng, Luigi Pistaferri, Kasper Roszbach}
}

\dateentry{Other}{
\href{https://www.diva-portal.org/smash/get/diva2:808180/FULLTEXT01.pdf}{\textbf{Long time integration of molecular dynamics at constant temperature with the symplectic Euler method (Master thesis, 2015)}, (\href{https://www.diva-portal.org/smash/get/diva2:808180/FULLTEXT01.pdf}{\underline{Link}})}
%I study the efficiency of the symplectic Euler method with an Ornstein-Uhlenbeck process of estimating coefficients in statistical physics systems by comparing it to other popular methods. I show that the method dominates alternatives for problems that require long-time integration to approximate the invariant measure.
}

\end{datetabular}






\iffalse
\section{Research papers}
\nocite{*}% Loads every entry from the attached .bib file
% Highlight takes three entries, given name, given name initials, and family name.
% If you have a middle initial, this call looks like:
% \highlightauthorname{Bob H.}{B. H.}{Smith}
\highlightauthorname{Jesper}{J}{Bojeryd} 
\begin{datetabular}{\datetabspace}
%printbibyear has been defined to only print entries from a given citation year.
\dateentry{2021}{\printbibyear{2021}}
\dateentry{2015}{\printbibyear{2015}}
\end{datetabular}
\fi

%\section{Presentations}


\section{Current positions}

\begin{datetabular}{\datetabspace}
%\dateentry{2022}{Norges bank (the central bank of Norway), Oslo, Norway

\dateentry{2022}{Guest Researcher, Norges Bank, Research Division

Guest Ph.D. student, Center of Monetary Policy and Financial Stability (at SU)
}
\dateentry{2021}{Visiting Graduate Student, Department of Economics, Stockholm University}
\dateentry{2020}{Research Scholar, UCLA Ziman Center for Real Estate}
\end{datetabular}

\section{Past positions}
\begin{datetabular}{\datetabspace}
    \dateentry{2023}{Dissertation Fellow, Federal Reserve Bank of San Francisco, Thomas J. Sargent D.F.}
    \dateentry{2022}{Ph.D. Intern, Norges Bank, Research Division}
    \dateentry{2021}{Visiting Research Fellow, Swedish House of Finance}
\end{datetabular}




\section{Relevant work experience}
\begin{datetabular}{\datetabspace}
\dateentry{2017--2018}{Research assistant, Swedish House of Finance for Marieke Bos}

\dateentry{2015--2017}{Research assistant, the Riksbank for Marieke Bos and Peter van Santen}
\end{datetabular}



\section{Grants, awards, and fellowships}

\begin{datetabular}{\datetabspace}
\dateentry{2023}{Best proseminar presentation 2022--2023 (UCLA); Thomas J. Sargent Dissertation Fellowship at the Federal Reserve Bank of San Francisco}

\dateentry{2022}{Handelsbankens forskningsstiftelser with Roine Vestman (\$190,000); % SEK 1,860,000 
Vinnova, Forskning on finansmarknader 2022--2024 with Vestman and co-authors (\$343,000); % SEK 3,366,000
Pandemic-related additional support (UCLA)}

\dateentry{2021}{Pandemic TA Award (UCLA), UCLA’s Lewis L. Clarke Graduate Fellowship Fund, Paulson Scholarship (UCLA)}

\dateentry{2020}{The Ziman Center’s UCLA Rosalinde and Arthur Gilbert Program in Real Estate, Finance and Urban Economics; UCLA’s European Studies Fellowship; UCLA’s Summer Graduate Fellowship; UCLA’s Graduate Summer Research Mentorship program}

\dateentry{2019}{Karlstad kommun: Frida och A O Ringqvists minnesfond}

\dateentry{2018}{Sverige-Amerika Stiftelsen: Fellow of Ernst O Ek’s stipendiefond}

\dateentry{2015}{The Fellows of KTH Scholarship Fund (Kamratstipendiefonden)}

\dateentry{2012}{Winner of EBEC Stockholm, and represented KTH at the Nordic finals}

\dateentry{2011}{Recipient of the Swedish Homeguard’s bronze medal}
\end{datetabular}

\section{Conferences and presentations}

\newcommand{\schmark}[0]{\textsuperscript{\dag}}
\newcommand{\comark}[0]{\textsuperscript{*}}

\begin{datetabular}{\datetabspace}

\dateentry{Scheduled}{ASSA 2024\comark}

\dateentry{2023}{North American Summer Meeting, UCLA, CSU Long Beach, the Riksbank\comark, Dolomiti Macro Meetings\comark, Swiss Winter Conference on Financial Intermediation (poster session)\comark, Arne Ryde Workshop\comark, Oslo Macro Conference\comark, Collegio Carlo Alberto\comark, Norwegian Business School\comark}

\dateentry{2022}{SOCAE, UCLA, Norges Bank, Greater Stockholm Macro Group\comark}
\end{datetabular}

\bigskip

\comark{} indicates presentations by co-authors of joint papers

\section{Teaching experience}
\begin{datetabular}{\datetabspace}
\dateentry{UCLA}{Teaching assistant \quad \href{https://www.jesperbojeryd.se/teaching}{Teaching evaluations can be found \underline{here}}
\begin{itemize}
    \item Statistics for Economists (Econ 41), Fall 2019 and Winter 2022
    \item Financial Markets and Financial Institutions (Econ 106M), Winter--Fall 2020
    \item Macroeconomic Theory (intermediate macro, Econ 102), Winter--Spring 2021
    \item Macroeconomics: Dynamics and Growth Theory (Ph.D. macroeconomics, Econ 202A), Fall 2021
    \item Corporate Finance (Econ 106F), Fall 2022
\end{itemize}\eatvspace}
\dateentry{KTH}{Teaching assistant
\begin{itemize}
    \item Probability theory and statistics
    \item Mathematical and numerical analysis
\end{itemize}\eatvspace}
\end{datetabular}


\section{Service}
%Reviewer for EAYE -- AM 2024, reviewer for SOCAE 2022, Ph.D. student mentoring
Reviewer for SOCAE 2022, Ph.D. student mentoring

\section{Other}
\begin{datetabular}{\datetabspace}
\dateentry{Languages}{English (fluent), Swedish (native)}
\dateentry{Computer skills}{Julia, Matlab, Stata, \LaTeX{} (advanced)\quad R, Github (intermediate)

Python, Java (rusty)\quad
See\quad\href{https://github.com/bojeryd91}{\faGithub\enspace\texttt{https://github.com/bojeryd91}}
}
\dateentry{Hobbies}{Trekking, running, beach volleyball, playing the trombone and the guitar}
\end{datetabular}





\end{document}
