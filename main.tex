\documentclass[letterpaper,10pt]{article}
\pagenumbering{gobble}
\usepackage[utf8]{inputenc}
\usepackage{titlesec, hyperref}

% Change behavior of and spacing around sections and paragraphs
\usepackage{titlesec}
\titleformat{\section}{\normalfont\large\bfseries}{\thesection}{1em}{}[{\titlerule[0.4pt]\vspace{3pt}}]
\titlespacing\section{0pt}{6pt}{0pt}
\titlespacing\paragraph{0pt}{6pt}{0pt}

\footskip=0pt
\setlength{\parindent}{0pt}

\usepackage[top=0.7in,left=1in,right=1in, bottom=1.5in,marginparwidth=1.1in]{geometry}

% A command with bold text and some additional space before
\newcommand{\bold}[1]{\ifhmode\hspace{6pt}\fi\textbf{#1:}}

\begin{document}
\begin{center}
{\Large \textbf{Curriculum vitae -- Jesper Böjeryd}}\\[3pt]
Born: 1991 \hspace{6pt} Nationality: Swedish \hspace{6pt} Email: bojeryd91@gmail.com \\
Phone: +1 (310) 775-3781 \hspace{6pt} Address: 1841 Veteran Avenue, Los Angeles, 90025, CA, USA \\
\url{www.jesperbojeryd.se}
\end{center}

\section*{Affiliations}
\bold{2021--22} Visiting Ph.D. student, Stockholm University, Department of Economics \bold{2020} Research scholar, UCLA Ziman Center for Real Estate; Research scholar, Swedish House of Finance (SSE) \bold{Since 2018} Ph.D. student, Department of Economics, UCLA.

\section*{Research}
\bold{The pass-through of monetary policy to automobile purchases of indebted Swedish households}\\
How household responds to monetary policy is important in understanding its distributional and aggregate effects. Previous research relies on imputed consumption for relatively short episodes. This paper studies the interaction of monetary policy and household indebtedness' effect on borrowers' purchasing decisions of automobiles, using a representative panel of Swedish households covering 21 years. %\textbf{Work in progress, draft coming soon.}

\bold{The heterogeneous effects of QE on corporate bonds and firm outcomes} with Adam Baybutt\\
Quantitative easing (QE) is becoming a standard tool of central banks to boost output and inflation in low-interest rate environments. However, evidence of real effects are scarce which makes it natural to question the efficacy of QE. We use corporate bond data to study the announcement effect of QE on corporate bond yields, and we find that announcements have heterogeneous effects, on average lowering corporate bond yields. The effects to real outcomes are small, if any. \textbf{Work in progress, draft on website}

\bold{The Housing Wealth Effect: Quasi-Experimental Evidence} with Dany Kessel, Bj\"orn Tyrefors, and Roine Vestman\\
Theory and empirical research disagree on how much homeowners respond in their consumption to changing house prices. By exploiting the surprise announcement of the continuation of Bromma Airport outside Stockholm, we estimate very small changes to car purchases to the 15\% fall in house prices the prolonged noise zone experienced, relative unaffected households living outside. \textbf{Work in progress, draft coming soon.}

\section*{Education}
\paragraph{2018--Exp. Jun 2024, Econ Ph.D. program}\hfill University of California, Los Angeles, USA\\
C.Phil, M.A., fields: Macroeconomics, Asset pricing\\
Committee: \href{https://www.leeohanian.com/}{\texttt{Lee Ohanian (Main advisor)}}, \href{https://sites.google.com/site/andyatkeson/}{\texttt{Andy Atkeson}}, \href{https://sites.google.com/site/kyleherkenhoff/}{\texttt{Kyle Herkenhoff}}, \href{https://sites.google.com/site/pierreolivierweill/}{\texttt{Pierre-Olivier Weill}}

\paragraph{2010--2015, Engineering physics}\hfill Royal Institute of Technology, Sweden\\
M.Sc.Eng, M.Sc, and B.Sc, concentration: Applied and computational mathematics\\
Master thesis: \href{https://www.diva-portal.org/smash/record.jsf?pid=diva2\%3A808180&dswid=-588}{\texttt{Long time integration of mole\-cular dynamics at constant  temperature with the symplectic Euler method}}, supervised by  Anders Szepessy

%The thesis studies a numerical algorithm to solve a class of problems in statistical physics. I investigate how to efficiently integrate functions over the whole state space for indefinite time. I studied articles on the topic, set up experiments, implemented methods, and evaluated and presented the results. The algorithm is applied to three different experiments and evaluated against alternative methods found in the literature.

\paragraph{Mar--Jun 2013, Exchange semester}\hfill KAIST, Republic of South Korea\\
Studied individual courses and wrote my bachelor thesis.


\paragraph{Spring 2010, Individual course}\hfill Karlstad University, Sweden\\
Elementary Algebra. This was a unique one-year project to encourage talented students to study college-level mathematics during my senior year in high school

\section*{Work experience}
\paragraph{2019--present, Teaching assistant}\hfill UCLA\\
I have TAed for the courses \emph{Statistics for economists}, \emph{Financial markets and financial institutions}, and the first quarter of PhD macroeconomics. Before every session I identify the concepts students are most unfamiliar with, and focus on those and connect them to their existing knowledge. I encourage a continuous dialogue by involving students in class and using OHs.

\paragraph{2015--2017, summer 2019, Research assistant}\hfill The Riksbank and Swedish House of Finance\\
As an RA for Marieke Bos and Peter van Santen, I have participated in every part of a research project. I sat in on out-of-house meetings to collect new data; planned, implemented, and presented data analysis using panel data econometrics; and discussed paper drafts and suggested improvements. I also held a one-day course in basic MATLAB programming for the Riksbank's forecasting division and provided general help in econometrics and programming to colleagues.

\paragraph{Periodically 2014--2015, Teaching assistant}
\hfill Royal Institute of Technology\\
Selection of courses: \emph{Probability theory and statistics}; \emph{Mathematical and numerical analysis}. I graded students based on finals, reports, and oral presentations, and tutored them by feedback and answering questions during lab sessions. We used mainly Matlab.

\section*{Language \& computer skills}
\bold{Native} Swedish \bold{Fluent} English \\
\bold{Advanced} MATLAB, Julia, Stata, \LaTeX; \bold{Intermediate} R; \bold{Beginner} Python, Java, Github

\section*{Awards \& fellowships}
\bold{2021} Pandemic TA Award (UCLA), UCLA’s Lewis L. Clarke Graduate Fellowship Fund, Paulson Scholarship (UCLA) \bold{2020} Ziman Center’s UCLA Rosalinde and Arthur Gilbert Program (UCLA), European Studies Fellowship (UCLA), Summer Graduate Fellowship (UCLA), Graduate Summer Research Mentorship program (UCLA) \bold{2019} Frida och A O Ringqvists minnesfond (Karlstad) \bold{2018} Ernst O Ek's stipendiefond (Sverige-Amerika Stiftelsen) \bold{2015}
The Fellows of KTH Scholarship Fund (Kamratstipendiefonden) \bold{2012} Winner of EBEC Stockholm and representative of KTH at the Nordic finals (BEST) \bold{2011} The Homeguard's bronze medal: \emph{Awarded for my diligent work as member of the Skaraborg Homeguard Marching Band.}

\section*{Extra-curricular activities} 
\bold{2015 and 2018} Volunteer at Mattecentrum as additional support in mathematics for high school students \bold{2000--2018} Musician, playing the trombone in several orchestras. \href{https://www.youtube.com/watch?v=nn6dDYi3AFg}{\texttt{Link to recording in pdf!}}
\iffalse
\iffalse
\emph{Several elected positions} in my division of KTH Student association, including:
\begin{lista2}
\emph{Treasurer} for a committee, \\
\emph{Producer} (head) for a play involving 100 people.  \bold{2011--2015}
\end{lista2}
\fi
\emph{Musician}, playing the trombone in several orchestras.% Lately mainly in symphonic orchestras.
\bold{2000--2018}

%\section{REFERENCES}
%\emph{Will be provided on request.}
%\iffalse
%{\setlength{\topsep}{-\parskip}%
%\setlength{\partopsep}{0pt}%
%\tabbing
%Anders Szepessy,\=\hspace*{6pt}\= former supervisor, \= +46 8 790 74 94,\hspace*{6pt} \= szepessy@kth.se\\
%Peter van Santen,\>\> former employer,    \> +46 8 787 05 69, \> \parbox[t][2em][t]{3.5cm}{peter.van.santen @riksbank.se}\\
%Marieke Bos, \>                  \> employer,          \> +46 73 573 74 61, \> Marieke.Bos@hhs.se
%\endtabbing}
%\fi
\fi
\end{document}